\documentclass{beamer}

\makeatletter
\def\input@path{{themes}}
\makeatother

\usetheme{custom}
\usecolortheme{custom-minimal}

% For code listings
\usepackage{listings}
\lstset{
  basicstyle=\ttfamily\small,
  keywordstyle=\color{blue},
  commentstyle=\color{green!60!black},
  stringstyle=\color{purple},
  breaklines=true,
  showstringspaces=false,
  frame=single
}

%----------------------------------------
% DOCUMENT BEGINS
%----------------------------------------
\title{Custom Theme LaTeX Presentation}
\subtitle{Creating Professional Slides with Beamer}
\author{Your Name}
\institute{Your Institution}
\date{\today}

\begin{document}

% Title slide
\begin{frame}[plain]
  \titlepage
\end{frame}

% Outline slide
\begin{frame}
  \frametitle{Presentation Outline}
  \tableofcontents
\end{frame}

%----------------------------------------
% SECTION 1: INTRODUCTION
%----------------------------------------
\section{Introduction}

\begin{frame}
  \frametitle{Introduction}
  
  Welcome to this custom-themed \LaTeX{} presentation!
  
  \begin{itemize}
    \item Beamer allows for professional presentations
    \item Custom themes enhance visual appeal
    \item This presentation demonstrates key features
  \end{itemize}
  
  \begin{alertblock}{Important Note}
    Customizing your presentation helps create a unique identity.
  \end{alertblock}
\end{frame}

\begin{frame}
  \frametitle{Benefits of \LaTeX{} Presentations}
  
  Why use \LaTeX{} for presentations?
  
  \begin{columns}
    \begin{column}{0.5\textwidth}
      \textbf{Technical Benefits}
      \begin{itemize}
        \item Precise control over formatting
        \item Excellent math typesetting
        \item Version control friendly
        \item Consistent output across platforms
      \end{itemize}
    \end{column}
    
    \begin{column}{0.5\textwidth}
      \textbf{Workflow Benefits}
      \begin{itemize}
        \item Focus on content, not formatting
        \item Reusable components
        \item Easily update and maintain
        \item PDF output is universal
      \end{itemize}
    \end{column}
  \end{columns}
\end{frame}

%----------------------------------------
% SECTION 2: KEY FEATURES
%----------------------------------------
\section{Key Features}

\begin{frame}
  \frametitle{Block Elements}
  
  Beamer provides various block elements:
  
  \begin{block}{Standard Block}
    This is a standard block element.
  \end{block}
  
  \begin{alertblock}{Alert Block}
    Important information goes here!
  \end{alertblock}
  
  \begin{exampleblock}{Example Block}
    This is an example to illustrate the concept.
  \end{exampleblock}
\end{frame}

\begin{frame}
  \frametitle{Mathematical Typesetting}
  
  \LaTeX{} excels at mathematical notation:
  
  \begin{align}
    E &= mc^2\\
    F &= G\frac{m_1 m_2}{r^2}\\
    \nabla \times \vec{B} &= \mu_0\vec{J} + \mu_0\epsilon_0\frac{\partial \vec{E}}{\partial t}
  \end{align}
  
  Maxwell's equations in differential form:
  \begin{align}
    \nabla \cdot \vec{E} &= \frac{\rho}{\epsilon_0}\\
    \nabla \cdot \vec{B} &= 0\\
    \nabla \times \vec{E} &= -\frac{\partial \vec{B}}{\partial t}\\
    \nabla \times \vec{B} &= \mu_0\vec{J} + \mu_0\epsilon_0\frac{\partial \vec{E}}{\partial t}
  \end{align}
\end{frame}

\begin{frame}[fragile]
  \frametitle{Code Listings}
  
  \begin{block}{Python Example}
    \begin{lstlisting}[language=Python]
def fibonacci(n):
    """Return the nth Fibonacci number."""
    if n <= 1:
        return n
    else:
        return fibonacci(n-1) + fibonacci(n-2)
        
# Calculate first 10 Fibonacci numbers
for i in range(10):
    print(fibonacci(i))
    \end{lstlisting}
  \end{block}
\end{frame}

%----------------------------------------
% SECTION 3: VISUAL ELEMENTS
%----------------------------------------
\section{Visual Elements}

\begin{frame}
  \frametitle{Tables}
  
  \begin{table}
    \centering
    \begin{tabular}{|l|c|r|}
      \hline
      \textbf{Left Aligned} & \textbf{Center} & \textbf{Right Aligned} \\
      \hline
      Item 1 & Value 1 & \$10.00 \\
      Item 2 & Value 2 & \$15.50 \\
      Item 3 & Value 3 & \$20.25 \\
      \hline
    \end{tabular}
    \caption{A simple table example}
  \end{table}
\end{frame}

\begin{frame}
  \frametitle{Graphics with TikZ}
  
  \begin{center}
    \begin{tikzpicture}
      % Axes
      \draw[thick,->] (0,0) -- (4.5,0) node[right] {$x$};
      \draw[thick,->] (0,0) -- (0,4.5) node[above] {$y$};
      
      % Function
      \draw[primarycolor, thick, domain=0:4, samples=100] plot (\x, {exp(-\x/2)*sin(deg(2*\x))});
      
      % Points
      \foreach \x in {1,2,3,4}
        \draw[fill=secondarycolor] (\x,{exp(-\x/2)*sin(deg(2*\x))}) circle (0.1);
      
      % Labels
      \node[below left] at (0,0) {$O$};
      \node[above] at (2, 1.5) {$f(x) = e^{-x/2}\sin(2x)$};
    \end{tikzpicture}
  \end{center}
\end{frame}

\begin{frame}
  \frametitle{Animation with Overlays}
  
  \begin{itemize}
    \item<1-> This point appears on the first slide
    \item<2-> This point appears on the second slide
    \item<3-> This point appears on the third slide
    \item<4-> This is the final point
  \end{itemize}
  
  \only<1>{
    \begin{alertblock}{First Step}
      Start with the basics.
    \end{alertblock}
  }
  
  \only<2>{
    \begin{alertblock}{Second Step}
      Build on the foundation.
    \end{alertblock}
  }
  
  \only<3->{
    \begin{alertblock}{Final Step}
      Complete the process.
    \end{alertblock}
  }
\end{frame}

%----------------------------------------
% SECTION 4: CUSTOMIZATION
%----------------------------------------
\section{Theme Customization}

\begin{frame}
  \frametitle{Customizing Your Theme}
  
  Key aspects to customize:
  
  \begin{enumerate}
    \item Colors
    \begin{itemize}
      \item Primary, secondary, and accent colors
      \item Background and text colors
    \end{itemize}
    
    \item Fonts
    \begin{itemize}
      \item Sans-serif for headlines
      \item Serif or sans-serif for body text
    \end{itemize}
    
    \item Layout elements
    \begin{itemize}
      \item Headers and footers
      \item Title page design
      \item Block styling
    \end{itemize}
  \end{enumerate}
\end{frame}

\begin{frame}
  \frametitle{Color Palette Options}
  
  \begin{columns}
    \begin{column}{0.5\textwidth}
      \begin{block}{Professional}
        \begin{itemize}
          \item Navy blue + Gray
          \item Dark green + Beige
          \item Burgundy + Light gray
        \end{itemize}
      \end{block}
      
      \begin{block}{Creative}
        \begin{itemize}
          \item Purple + Yellow
          \item Teal + Coral
          \item Orange + Blue
        \end{itemize}
      \end{block}
    \end{column}
    
    \begin{column}{0.5\textwidth}
      \begin{block}{Academic}
        \begin{itemize}
          \item University colors
          \item Subdued blue + White
          \item Forest green + Ivory
        \end{itemize}
      \end{block}
      
      \begin{block}{Corporate}
        \begin{itemize}
          \item Brand colors
          \item Black + Accent color
          \item Grayscale + Highlight
        \end{itemize}
      \end{block}
    \end{column}
  \end{columns}
\end{frame}

%----------------------------------------
% SECTION 5: CONCLUSION
%----------------------------------------
\section{Conclusion}

\begin{frame}
  \frametitle{Summary}
  
  \begin{block}{Key Takeaways}
    \begin{itemize}
      \item \LaTeX{} Beamer provides professional presentation capabilities
      \item Custom themes allow for unique, branded presentations
      \item Advanced features include:
        \begin{itemize}
          \item Mathematical typesetting
          \item Code listings
          \item Tables and figures
          \item Animations and overlays
        \end{itemize}
      \item Perfect for academic and technical presentations
    \end{itemize}
  \end{block}
\end{frame}

\begin{frame}[plain]
  \centering
  \vspace{2cm}
  {\huge Thank You!}
  
  \vspace{1cm}
  {\Large Questions?}
  
  \vspace{2cm}
  {\ttfamily your.email@example.com}
\end{frame}

\end{document}